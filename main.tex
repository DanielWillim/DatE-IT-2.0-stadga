\documentclass{article}
\usepackage[utf8]{inputenc}

\usepackage{parskip}

\usepackage{comment}

\title{Styrdokument för DatE-IT}
\author{Daniel Willim}
\date{January 2022}

\begin{document}

\maketitle

\begin{comment}
TODO 
- Fixa så att alla sektion pekar på detta dokument
- E ska fastslå budget men ska inte kunna ändra saker - delegerat detta mandat till oss
\end{comment}

\section{Allmänt}
\subsection{Ändamål}
Detta dokument gäller för DatE-IT samt dess verksamhet. Det fastställs av de tre sektionsstyrelserna separat. Ändringar görs på förslag av DatE-IT-styrelsen eller någon av sektionsstyrelsena.

\subsection{Definitioner}
DatE-IT är en arbetsmarknadsmässa som arrangeras som ett samarbete mellan Datateknologsektionen, D (857209-7080), Teknologsektionen Informationsteknik, IT (857209-9524) och Elektroteknologsektionen, E (857202-2013). Hädanefter kommer sektionerna referera till dessa tre sektioner och sektionsstyrelserna refererar till styrelsen vid respektive sektion.

Styrelsen refererar till DatE-IT-styrelsen enligt kap. ** och kommittén refererar till DatE-IT-kommittén enligt kap. **.

\subsection{Verksamhetsår}
DatE-ITs verksamhetsår är 1 April - 31 Mars. 

\subsection{Ekonomi}
DatE-ITs ekonomi hanteras av Elektroteknologsektionen. 

\subsubsection{Fördelning av vinst och förlust}
DatE-ITs resultat i slutet av verksamhetsåret fördelas lika mellan samtliga av sektionerna.

\section{DatE-IT kommittén}
\subsection{Medlemmar}
Kommitten består av nio ledamöter varav en projektledare och en ekonomiskt ansvarig ledamot.

\subsection{Mandatperiod}
Mandatperioden för kommittén är densamma som verksamhetsåret.

\subsection{Inval}
Kommittén väljs i första hand in på vardera sektions sektionsmöte i läsperiod 3.

DatE-ITs valberedning (se kap **) har rätt att nominera ledamöter till kommittén.

Varje sektion har rätt att tillsätta upp till tre ledamöter. 

I den händelsen att hela eller delar av kommitten inte tillsätts vid respektive sektions sektionsmöte i läsperiod 3 så ansvarar styrelsen i samarbete med valberedningen för inval av dessa. Dessa beslut ska sedan informeras om på vardera sektions sektionsmöte. 

Projektledaren väljs bland kommitténs ledamöter av styrelsen.

\subsection{Ansvar och skyldigheter}
\begin{itemize}
    \item Kommittén ansvarar för att arrangera en arbetsmarknadsmässa för sektionernas medlemmar. 
    \item Kommittén ansvarar för att lägga ett förslag på budget och verksamhetsplan till styrelsen till sista styrelsemötet innan sommaren
    \item DatE-ITs projektledare ska föredra kommitténs verksamhet under styrelsemötena.
\end{itemize}

\subsection{Rättigheter}
Kommittén har rätten att belasta budgeten med möteskostnader, representationskläder, teambuilding, resor samt rekryteringskostnader. Dessa förmåner ska godkännas av styrelsen genom budgeten.

\section{DatE-IT Styrelsen}
\subsection{Medlemmar}
Styrelsen består av följande medlemmar:
\begin{itemize}
    \item En representant från vardera sektionsstyrelse.
    \item Tre övriga ledamöter varav en ordförande.
    \item Projektledare i kommittén.
\end{itemize}

\subsection{Sammanträden}
Styrelsen ska sammanträda minst en gång per läsperiod. Styrelsen är endast beslutsmässig då minst 4 ledamöter deltar.

Inför sammanträde ska kallelse skickas ut minst två veckor i förväg för att vara beslutsmässiga. Justerat protokoll ska publiceras senast 2 veckor efter mötet.

Kallelse och protokoll ska skickas till samtliga styrelseledamöter och sektionsstyrelser.

\subsubsection{Omröstning}
\begin{itemize}
    \item Representanterna från sektionsstyrelserna har två röster var.
    \item De övriga ledamöterna har en röst var.
    \item DatE-ITs projektledaren har inga röster.
\end{itemize}

Vid oavgjord röstning bifalls den mening som biträds av flest representanter från sektionsstyrelser.

Vid projektledares frånvaro ska en annan ledamot i kommittén närvara istället. Vid frånvaro för sektionsstyrelserepresentant ska annan ledamot i styrelsen närvara istället. 

\subsection{Mandatperiod}
Representanterna från sektionsstyrelserna väljs tills dess att en ny representant utses av sektionsstyrelsen.

De övriga ledamöterna väljs för två år. De övriga ledamötenas mandatperioder ska vara överlappande så att inte alla byts ut samtidigt.

\subsubsection{Avsägelse}
Ledamot i styrelsen kan avsäga sig sin plats genom att skriftligen meddela styrelsen. Avsägelsen ska sedan hanteras på nästkommande styrelsemöte där en ny person ska väljas enligt \ref{styrele-val} för resterande av ledamotens mandatperiod.

\subsection{Inval} \label{styrelse-val}
Representanterna från sektionsstyrelse väljs av vardera sektionsstyrelse ur sektionsstyrelsens.

De övriga ledamöterna väljs av styrelsen med $\frac{2}{3}$ absolut majoritet vid sista styrelsemötet innan verksamhetsårets slut.

Styrelsens ordförande för det kommande verksamhetsåret väljs bland de övriga ledamöterna internt av styrelsen vid verksamhetsårets första möte.

Samma person får inte väljas till representant från sektionsstyrelse och övrig ledamot samtidigt.

\subsection{Ansvar och Skyldigheter}
Styrelsen ansvarar för:
\begin{itemize}
    \item Stötta kommitén i den strategiska utveckling av DatE-IT .
    \item Godkänna budget och verksamhetsplan på förslag från kommittén.
    \item Välja valberedning för kommittén (enligt punkt om DatE-ITs valberedning)
    \item Efterrekrytering av kommittén (enligt punkt om kommitténs inval)
    \item Tillsätta övriga ledamöter i styrelsen
    \item Presentera ekonomin och verksamhet för sektionsmöten regelbundet? – kommittén och styrelsen
    \item Föredra stora beslut för sektonsmöterna.
    \item Informera sektionerna om relevanta beslut och verksamheten.
\end{itemize}

\subsection{Rättigheter}
Styrelsen äger rätt att belasta budgeten med möteskostnader för styrelsemötena med vad som anses lämpligt.

DatE-ITs styrelse äger rätt att skapa och ta bort ytterligare styrdokument utöver detta dokument. Dessa styrdokument ska gälla för DatE-ITs verksamhet och får inte strida mot detta dokument.

\section{DatE-IT valberedning}
DatE-ITs valberedning väljs av styrelsen. 

\subsection{Ansvar}
Valberedningen ansvarar för att nominera medlemmar från sektionerna till kommittén.

\section{Ändrings- och tolkingsfrågor}
\subsection{Ändring}
Ändring av detta styrdokument kan endast göras av styrelsen. För att vara giltig måste ändringen antas med $\frac{2}{3}$ absolut majoritet. Förslag på förändring måste ha skickats ut i kallelse senast två veckor innan beslutet.

\subsection{Tolkning}
Uppstår tolkningstvist om dessa stadgars tolkning, tolkas stadgan av Chalmers studentkårs inspektor.

\section{Revision}
Alla handlingar ska vara samtliga sektioner tillgängliga för revision.

Revision sköts av Elektroteknologsektionens lekmannarevisorer.

\section{Misstroendeförklaring}
Nedan listade personer äger rätt att yrka på en misstroendeförklaring av medlem i kommittén eller styrelsen genom att skriftligt kontakta alla 3 sektionssyrelser. 
\begin{itemize}
    \item Sektionssmedlem
    \item Medlem i styrelsen eller kommittén
    \item Sektionsstyrelseledamot
    \item Lekmannarevisor vid någon av sektionerna
    \item Inspektor
\end{itemize}

Detta ska sedan lyftas på samtliga sektionsstyrelsemöten inom 3 veckor och verkställs givet att minst 2 av 3 sektionsstyrelser bifaller yrkandet.

\begin{comment}
#Övriga tankar från styrIT:
Vem lägger och godkänner budget? Som det ser ut nu i dokumentet ska styrelsen godkänna budgeten, men de har också utgifter. Det känns inte bra att de både skriver och lägger budgeten. Att den går genom sektionsmötena är ju en lösning, men hela poängen med det här var att slippa det. En annan lösning är att spika ett maxbelopp i styrdokumentet för styrelsens möteskostnader.
Är verksamhetsplan en grej? I så fall har den liknande problem som budgeten. Om det inte är en grej, kanske det borde bli

Kanske bättre om två ledamöter sitter på en 2-årsbasis (men olika invalsår) och en på en 1-årsbasis?
\end{comment}

\end{document}
