\documentclass{article}
\usepackage[utf8]{inputenc}

\usepackage{parskip}

\title{Styrdokument för DatE-IT}
\author{Daniel Willim}
\date{January 2022}

\begin{document}

\maketitle

\section{Allmänt}
\subsection{Ändamål}
Detta dokument gäller för DatE-IT samt dess verksamhet. Det fastställs av de tre sektionsstyrelserna separat. Ändringar görs på förslag av DatE-IT-styrelsen eller någon av sektionsstyrelsena.

DatE-ITs styrelse äger rätt att skapa och ta bort ytterligare styrdokument utöver detta dokument. Dessa styrdokument ska gälla för DatE-ITs verksamhet och får inte strida mot detta dokument.

\subsection{Definitioner}
DatE-IT arrangeras som ett samarbete mellan Datateknologsektionen, D (857209-7080), Teknologsektionen Informationsteknik, IT (857209-9524) och Elektroteknologsektionen, E (857202-2013). Hädanefter kommer sektionerna referera till dessa tre sektioner och sektionsstyrelserna refererar till styrelsen vid respektive sektion.
Styrelsen refererar till DatE-IT-styrelsen enligt kap. ** och kommittén refererar till DatE-IT-kommittén enligt kap. **.

\subsection{Verksamhetsår}
DatE-ITs verksamhetsår är densamma som Elektroteknologsektionens verksamhetsår.

\section{DatE-IT kommittén}
\subsection{Medlemmar}
Kommitten består av nio ledamöter varav en projektledare.

\subsection{Mandatperiod}
Mandatperioden för kommittén är densamma som verksamhetsåret.

\subsection{Inval}
Kommittén väljs i första hand in på vardera sektions sektionsmöte i läsperiod 3.

DatE-ITs valberedning (se kap **) har rätt att nominera ledamöter till Kommittén.

Varje sektion har rätt att tillsätta upp till tre ledamöter. 

I den händelsen att hela eller delar av kommitten inte tillsätts vid respektive sektions sektionsmöte i läsperiod 3 så ansvarar styrelsen för inval av dessa.

Projektledaren väljs bland kommitténs ledamöter av styrelsen.


\subsection{Ansvar och skyldigheter}
\begin{itemize}
    \item Kommittén ansvarar för att arrangera en arbetsmarknadsmässa för sektionernas medlemmar. 
    \item DatE-ITs projektledare ska närvara på styrelsens möten där verksamanehten informaers om. annars .
    \item Kommittén ansvarar för att lägga ett förslag på budget till styrelsen till sista styrelsemötet innan sommaren
    \item VP till styrelse
\end{itemize}

\subsection{Rättigheter}
Kommittén har rätten att belasta budgeten med möteskostnader, representationskläder, teambuilding, resor samt rekryteringskostnader. Dessa förmåner ska godkännas av styrelsen genom budgeten.

\section{Fördelning av vinst och förlust}
DatE-ITs resultat i slutet av verksamhetsåret fördelas lika mellan samtliga av sektionerna.

\section{DatE-IT Styrelsen}
\subsection{Medlemmar}
Styrelsen består av följande medlemmar:
\begin{itemize}
    \item En representant från vardera sektionsstyrelse.
    \item Tre övriga ledamöter varav en ordförande.
    \item Projektledare i kommittén.
\end{itemize}

\subsection{Omröstning}
\begin{itemize}
    \item Representanterna från sektionsstyrelserna har två röster var.
    \item De övriga ledamöterna har en röst var.
    \item Projektledaren har inga röster.
\end{itemize}

Samma person får inte väljas till representant från sektionsstyrelse och övrig ledamot samtidigt.

Vid oavgjord röstning bifalls den mening som biträds av flest representanter från sektionsstyrelser.

Vid projektledares frånvaro ska en annan ledamot i kommittén närvara istället.

\subsection{Mandatperiod}
Representanterna från sektionsstyrelserna väljs tills dess att en ny representant utses av sektionsstyrelsen.

De övriga ledamöterna väljs för två år.

\subsection{Val}
Representanterna från sektionsstyrelse väljs av vardera sektionsstyrelse ur sektionsstyrelsens.

De övriga ledamöterna väljs av styrelsen med minst 2 / 3 av antalet absoluta röster. Innan en ledamot vars mandatperiod tar slut tar slut

Styrelsens ordförande väljs bland de övriga ledamöterna internt av styrelsen för varje verksamhetsåret.

\subsection{Ansvar och Skyldigheter}
Styrelsen ansvarar för:
\begin{itemize}
    \item Stötta kommitén i den strategiska utveckling av DatE-IT .
    \item Godkänna budget.
    \item Välja valberedning för kommittén (enligt punkt om DatE-ITs valberedning)
    \item Efterrekrytering av kommittén (enligt punkt om kommitténs inval)
    \item Tillsätta övriga ledamöter i styrelsen
    \item Presentera ekonomin och verksamhet för sektionsmöten regelbundet? – kommittén och styrelsen
    \item Ansvarar för DatE-IT gentemot sektonsmöterna på respektive sektioner.
    \item Informera sektionerna om relevanta beslut och verksamhet
\end{itemize}

\subsection{Rättigheter}
Styrelsen äger rätt att belasta budgeten med möteskostnader för styrelsemötena med vad som anses lämpligt.

\subsection{Sammanträden}
Styrelsen ska sammanträda minst en gång per läsperiod.

Inför sammanträde ska kallelse skickas ut minst två veckor i förväg  för att vara beslutsmässiga.

Kallelse till sammanträden ska skickas till samtliga styrelseledamöter och sektionsstyrelser.

\section{DatE-IT valberedning}
DatE-ITs valberedning väljs av styrelsen. 

\subsection{Ansvar}
Valberedningen ansvarar för att nominera medlemmar från sektionerna till kommittén.

\section{Ändrings- och tolkingsfrågor}
\subsection{Ändring}
Ändring av detta styrdokument kan endast göras av styrelsen. För att vara giltig måste ändringen antas med minst 2 / 3 av antalet avlagda röster. Förslag på förändring måste ha skickats ut i kallelse senast två veckor innan beslutet.

\subsection{Tolkning}
Uppstår tolkningstvist om dessa stadgars tolkning, tolkas stadgan av Chalmers studentkårs inspektor.

\section{Revision}
Alla handlingar ska vara tillgängliga för samtliga sektioner för revision.

Revision sköts av Elektroteknologsektionens lekmannarevisorer.


\end{document}
